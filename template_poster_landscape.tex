% Imperial College London Poster (Landscape)
% LaTeX Template
% Version 1.0 (February 22, 2024)
%
% For current versions and to report
% issues, please see:
% https://github.com/ImperialCollegeLondon/imperial_latex_templates
%
%!TEX program = xelatex
% Note: this template must be compiled with XeLaTeX rather than PDFLaTeX
% due to the custom fonts used. The line above should ensure this happens
% automatically, but if it doesn't, your LaTeX editor should have a simple toggle
% to switch to using XeLaTeX.
%
% © Imperial College London, 2024. This template, including logo and fonts, is 
% for use of Imperial staff and students only for university business. All rights 
% reserved to the copyright owners.
%
%%%%%%%%%%%%%%%%%%%%%%%%%%%%%%%%%%%%%%%%%

%----------------------------------------------------------------------------------------
%	CLASS, PACKAGES AND OTHER DOCUMENT CONFIGURATIONS
%----------------------------------------------------------------------------------------

\documentclass[
	landscape, % Landscape page orientation
	%print, % Uncomment to convert colors to CMYK for printing purposes
	%a1papersize, % Uncomment to use an A1 paper size instead of the default A0 paper size
]{ImperialPoster}
\usepackage[
  backend=biber,
  style=numeric
]{biblatex}
\usepackage{amssymb, amsmath, amsthm}
\usepackage{amsfonts}
\usepackage{mathtools}
\usepackage{graphicx, psfrag, setspace, subfigure, gensymb}
\usepackage{color}
\usepackage[justification=centering]{caption}
\usepackage{multicol} % This is so we can have multiple columns of text side-by-side
\columnsep=100pt % This is the amount of white space between the columns in the poster
\columnseprule=3pt % This is the thickness of the black line between the columns in the poster

\newtheorem{thm}[equation]{Theorem}  
\addbibresource{bibliography.bib}

%----------------------------------------------------------------------------------------
%	POSTER INFORMATION
%----------------------------------------------------------------------------------------

\postertitle{Homological mirror symmetry for some orbifold del Pezzo surfaces and the special McKay correspondence} % The poster title, can be split across multiple lines manually with \\

% Contributor/author names, can be split across multiple lines manually with \\
% Add affiliations using the \affiliation{} command
\posterauthors{Bogdan Simeonov (Imperial College London,as part of the London School of Geometry and Number Theory)}

% Command to output coauthor logos to the right of the Imperial logo in the header of the poster
% If no coauthor logos are required, remove or comment out this command

\coauthorlogos{
	\coauthorlogo[4cm]{UKRI_logo.png} % Use the optional parameter to specify a height for the logo
	\coauthorlogo[4cm]{lsgntlogo.png} % Use the optional parameter to specify a height for the logo
	%\coauthorlogo{Grey_16-9.pdf} % Not specifying the optional parameter will make the logo the same height as the Imperial logo
}
%----------------------------------------------------------------------------------------

\begin{document}

\titlesection % Output the title section, automatically populated using the information in the POSTER INFORMATION block above

\medskip % Vertical whitespace

\begin{multicols}{3} % Start the four-column layout
	
	\subsection{\Large Introduction and main result}
	
	We study a class of orbifold del Pezzo surfaces with a single $\frac{1}{k}(1,1)$ point, thought of as the family of hypersurfaces $X_{k+1}\subset \mathbb{P}(1,1,1,k)$. Their derived category has a particularly nice semiorthogonal decomposition due to the special McKay correspondence (Ueda-Ishii, Rota-Gugiatti):
	$$\mathcal{D}^b(\mathcal{X})=\langle e_2, \dots, e_{k-1}, \Phi D^b(Y)\rangle$$ where $Y$ is the resolution of the underlying singular coarse space of $\mathcal{X}$ and the $e_i$ are skyscraper sheaves supported at the orbifold point, with some extra representation-theoretic data.

	\begin{thm}There is a mirror LG model $W:M^0\rightarrow \mathbb{C}$ whose generic fiber is a hyperelliptic curve of genus $\frac{k+1}{2}$ and an explicit mirror map associating to a complexified Kahler class $[B+i\omega]$ an element $\mathcal{X}$ in the family of del Pezzo surfaces such that $$\mathcal{D}^b(\mathcal{X})\simeq \mathrm{Lag}_{vc}(M^0,W; [B+i\omega])$$
	Moreover, there is a complex submanifold $M^{in}\subset M^0$ which is a Stein subdomain when $M^0$ is exact which is a fibration of punctured elliptic curves for which $$D^b(Y)\simeq \mathrm{Lag}_{vc}(M^{in}, W|_{M^{in}}; [B+i\omega]|_{M^{in}})$$
	As such, the fully faithful functor $\Phi:\mathcal{D}^b(Y)\rightarrow \mathcal{D}^b(\mathcal{X})$ can be thought of as a non-exact analogue of the forward stopped inclusion functors of Sylvan/Ganatra-Pardon-Shende.
	\end{thm}


	
	\subsection{\Large Previous results}
	
	Our paper builds and generalizes techniques of Auroux-Katzarkov-Orlov (who proved HMS for smooth del Pezzo surfaces, as well as weighted projective planes), as well as Hacking-Keating, who proved HMS for smooth log CY surfaces at the large volume limit by constructing an abstract Lefschetz fibration. In particular, the manifold $M^{in}$ serves as an explicit realization of Hacking-Keating's abstract Lefschetz fibration, and moreover we are able to consider mirror symmetry over a large open subset of the space of complex structures rather than just the large volume limit point.
	

		
	%\columnbreak % Switch to the next column
	
	%----------------------------------------------------------------------------------------
	%	SECOND COLUMN
	%----------------------------------------------------------------------------------------
	
	\subsection{\Large Strategy of proof}
	
	The mirror manifold $M^0$ is a partial compactification of $(\mathbb{C}^\times)^2$ on which the potential function looks like $$W|_{(\mathbb{C}^\times)^2}=\frac{\prod_{i=1}^{k+1} (y+q_i)}{xy}+x+(\tau_1y+\dots+\tau_{\frac{k-1}{2}}y^{\frac{k-1}{2}})$$
	\smalltext{Remark: The parameters $q_i$ are Novikov parameters associated to classes in $H_2(\mathcal{X})$ and the $\tau_i$ are parameters associated to twisted orbifold sectors. As such, they depend on the symplectic structure of $\mathcal{X}$ and determine the complex structure of $M^0$. Since we are for now concerned with $\mathcal{X}$ as B-model, we can more or less freely set the values of these parameters.}
	We construct a deformation of the potential depending on an extra parameter $s$ such that for $s\approx 0$ the critical values of $W_s$ are distributed as in the following figure:
	
	\begin{figure}[H] 
		\centering
		\includegraphics[scale=3]{criticalvaluesandpaths.pdf} \caption{ Critical values in the case $k=15, q_i=1, \tau_1=1, \tau_i>0, s\approx 0$. At the origin, there is a fiber containing $k$ holomorphic spheres.}
	\end{figure}
	A careful choice of vanishing paths (differing from the one depicted above by passing to the left dual of the paths going to the outer purple critical values) will give us a collection of Lagrangian vanishing cycles, whose topological type we compute on the general fiber. We get a semiorthogonal decomposition $$\langle \color{purple}\tilde{L}_{2},\dots, \tilde{L}_{k-1}, \color{blue}P_0, \tilde{P},P_1, \color{orange}B_1, \dots, B_{k+1}\rangle$$ which we claim is mirror to the full exceptional collection $$\langle e_2, \dots, e_{k-1}, \Phi \mathcal{O}, \Phi \mathcal{T}(-H),\Phi \mathcal{O}(H), \Phi\mathcal{O}_{B_1},\dots, \Phi\mathcal{O}_{B_{k+1}}\rangle$$
	To prove this, we choose grading data so that everything is concentrated in degree $0$ and hence the $A_\infty$ structure reduces to a computation of the composition law in both categories.



	\subsection{\Large Resolving orbifold points on the B-side as handle attachment surgery on the A-side}
	
	To recover the mirror to $Y$ from the mirror to $\mathcal{X}$, one passes to a codimension $0$ complex submanifold $M^{in}\subset M^0$. This can be constructed as $M^{in}:=\{m \in M|\, |W(m)|^2\leq R,|y(m)|^2\leq R\}$ for a sufficiently large $R$. This is essentially restricting the LG model to a smaller disk, ignoring the outer critical values, but also making the generic fiber smaller by removing some handles, which can be visualized as:

	\begin{figure}[H] 
		\centering
		\includegraphics[scale=2]{HyperellipticHandles.pdf} % Figure image
		\caption{ A punctured genus $\frac{k+1}{2}$ hyperelliptic curve, thought of as a twice-punctured elliptic curve with $k-2$ handle attachments.}
	\end{figure}

	
	%\subsection{Affiliations}
	
	% Affiliation institutions, the first parameter is the affiliation number (matching the author list) and the second is the institution
	%\affiliationentry{1}{Department Name, Imperial College London}\\
	%\affiliationentry{2}{University Name}\\
	%\affiliationentry{3}{Institute Name}
	\smalltext{
	\subsection{References}
	\begin{enumerate}
		\item Sylvan, Zachary. {\ImperialSansLight On partially wrapped Fukaya categories.}
		\item Auroux, Denis; Katzarkov, Ludmil; Orlov, Dmitri. {\ImperialSansLight Mirror \,symmetry\, for\, Del \,Pezzo\, surfaces:\, Vanishing\, cycles\, and\, coherent\, sheaves}
		\item Auroux, Denis; Katzarkov, Ludmil; Orlov, Dmitri. {\ImperialSansLight Mirror\, symmetry\, for\, weighted\, projective\, planes\, and\, their\, noncommutative\, deformations.}
		\item GPS
		\item Rota
		\item ishii
		\item Hacking-Keating. {\ImperialSansLight This is light text.}
	\end{enumerate}}
	
	
	\nocite{*}
	\printbibliography
	%\subsection{Funders}
	
	%\includegraphics[width=0.55\linewidth]{UKRI_logo.png}
	%\hfill \hfill\includegraphics[width=0.55\linewidth]{lsgntlogo.png} % Side by side funder logos
	
%----------------------------------------------------------------------------------------

\end{multicols}

\end{document}
